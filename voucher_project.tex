\documentclass[12pt]{article}

\usepackage[style=authoryear,uniquename=false,backend=biber]{biblatex}
\usepackage{pgf}
\bibliography{tombib}
\usepackage{amsmath}

\begin{document}

\title{Rental assistance vs public housing}
\author{Tom Davidoff}
\maketitle

Some thoughts on the problem:

\begin{itemize}
	\item Not sure how to handle First Nations here, I will keep general, but that's obviously a big issue.
	\item Second welfare theorem: give cash, not housing units or even housing price cuts
		\begin{itemize}
			\item Powerful mechanism around Vancouver: CAC/density bonusing (see \textcite{ElmendorfShanske} and \textcite{Davidoffauction}).
			\item Idea: set targets to densify single family areas with predefined allowable density, auction quantities
			\item Current system of CACs and Density Bonusing accomplishes this if cities set quantity targets and set pricing to meet those targets
			\item Potentially huge cash amounts, billions already in Burnaby, 30\% (??) of Vancouver city budget at peak
			\item Note most jurisdictions around North America can't charge more than cost of on-site improvements
			\item So affordability requirement in a new project in lieu of a CAC yields an auction for a very large housing prize
			\item Opportunity cost of public land very high
		\end{itemize}
	\item Why do the welfare theorems not hold here?
		\begin{itemize}
			\item Some people presumably face high costs of leaving high cost locations
				\begin{itemize}
					\item Family/social attachment (\textcite{BlanchardKatz}-- real wages don't converge, but unemployment does; \textcite{GlaeserAustinSummers}, starting to see divergence.
					\item Canada presumably more stickiness
					\item So need adjustment for local costs, creating a mobility (exit the region) incentive problem, so welfare theorems are out the window
				\end{itemize}
			\item Parent-child allocation of cash aid
				\begin{itemize}
					\item Child-centric consumption less work disincentive
					\item Free cash moral hazard spending problem
					\item Not easily solved as rent assistance must be $>$ otherwise optimal rent expenditure to bind
				\end{itemize}
			\item Concentration of poverty
				\begin{itemize}
					\item Not a big deal here unlike the US (Somerville-Reis there are more desirable school catchments in Vancouver)
				\end{itemize}
			\item Sub-standard private housing creating health problems (??)
			\item Homelessness -- need for on-site help for at-risk populations?
		\end{itemize}
	\item Rental assistance
		\begin{itemize}
			\item Moral hazard on rent if formulaic to income: with homeowner grant capped for many would be a ``live in BC housing allowance''
				\begin{itemize}
					\item \textcite{EriksenRoss} 30\% of income so landlords bump rent and capture upside
				\end{itemize}
			\item Moral hazard on unit choice to parents. US Section 8 Voucher experience: 
				\begin{itemize}
					\item \textcite{PalmerChettyetc} Voucher recipients make dominated housing choices that are improved with information and outreach, builds on MTO.
				\end{itemize}
		\end{itemize}
	\item LIHTC and crowdout
		\begin{itemize}
			\item Heterogeneous impacts on surrounding housing prices based on difference with existing units \textcite{DiamondMcQuade}, \textcite{BaumSnowMarion}
			\item \textcite{SinaiWaldfogelcrowdout}:: more crowdout of private housing units from public or LIHTC housing than from housing vouchers
		\end{itemize}
	\item Difficulty of purpose built rental -- tax reform and low interest rates undersold as a reason not enough PBR (Canada similar situation to US) (\textcite{Poterba92}).
	\item Federal government putting a lot of money into rental, might be better done as public ownership, see my note \textcite{Davidoffcmhc}. Why lend at 1.75\% rent/price when equity partner developers insist on 4.5\% rent/price to do development deals? Government could buy from developers at 3\%, get more built, and much better returns for taxpayers (up to risk, which seems quite limited and risk of low rents is a win).
\end{itemize}

\printbibliography

\end{document}
